\documentclass[a4paper,10pt]{article}
\usepackage[utf8]{inputenc}
\usepackage{geometry}
\geometry{a4paper, margin=0.7in}
\usepackage{enumitem}
\usepackage{hyperref}
\usepackage{graphicx} % For including images
\usepackage{fontspec} % For using system fonts
\usepackage{paracol} % For creating multi-column layouts
\usepackage{fontawesome5} % For using icons
\usepackage{array} % For using m{} column type
\usepackage[most]{tcolorbox} % For creating colored boxes
\usepackage{xcolor} % For changing text color
\usepackage{tikz} % For drawing shapes

% Define ACCENT COLOR
\definecolor{accentcolor}{RGB}{0, 114, 178}

% Set the main font to Helvetica
\setmainfont{Helvetica}

% Custom commands for personal information
\newcommand{\name}[2]{\textcolor{accentcolor}{\textbf{\LARGE #1 #2}}}
\newcommand{\position}[1]{{\large #1}}
\newcommand{\address}[1]{{#1}}
\newcommand{\mobile}[1]{{#1}}
\newcommand{\email}[1]{{\href{mailto:#1}{#1}}}
\newcommand{\homepage}[1]{{\href{http://#1}{#1}}}
\newcommand{\github}[1]{{\href{http://github.com/#1}{#1}}}
\newcommand{\linkedin}[1]{{\href{http://linkedin.com/in/#1}{#1}}}
\newcommand{\dateofbirth}[1]{{#1}}
\newcommand{\residence}[1]{{#1}}
\newcommand{\citizenship}[1]{{#1}}

\pagestyle{empty} % Disable headers and footers

\begin{document}

    \sloppy % Allow LaTeX to adjust spacing to avoid overfull boxes

    \columnratio{0.7, 0.3}
    \begin{paracol}{2}
        \begin{nthcolumn}{0}
            \raggedright % Aligns the content of the minipage to the left
            \vspace{0pt} % Aligns the top of the minipage with the top of the image
            \name{John}{Doe} \\
            \vspace{0.25cm} % Add vertical space between name and position
            \position{AI Engineer} \\
            \vspace{0.5cm}
            I am a passionate and dedicated AI specialist with a strong background in research and development. My expertise lies in developing innovative AI solutions to solve complex problems and drive technological advancements. With a solid foundation in machine learning, deep learning, and data analysis, I am committed to pushing the boundaries of what is possible with artificial intelligence. I am eager to contribute my skills and knowledge to a forward-thinking organization that values cutting-edge research and development in the AI domain.
        \end{nthcolumn}
        \begin{nthcolumn}{1}
            \vspace{0pt} % Aligns the top of the minipage with the top of the text

            % USE THIS FOR already rounded picture
            \begin{tikzpicture}
                \clip (0,0) circle (0.12\textwidth);
                \node[anchor=center] at (0,0) {\includegraphics[width=0.2\textwidth]{profile-round.png}};
                \draw[accentcolor, line width=4pt] (0,0) circle (0.11\textwidth); % Draw a accent color border around the image
            \end{tikzpicture}  
            
        \end{nthcolumn}
    \end{paracol}

    \vspace{0.5cm} % Add some vertical space before the horizontal line
    \makebox[0.94\textwidth][c]{\rule{\textwidth}{1pt}} % Center the horizontal line and set thickness

    \vspace{0.5cm} % Add some vertical space before the columns

    \columnratio{0.6, 0.4}
    \begin{paracol}{2}
        \setlength{\columnsep}{20pt} % Set the column separator width
        \setlength{\columnseprule}{1pt} % Add a vertical rule between the columns

        \begin{nthcolumn}{0}
            \raggedright % Align the content to the left
            \textcolor{accentcolor}{\textbf{\large {Work Experience}} }\\[0.2cm]

% ----------------------------
% Add main job entries as follows
\textbf{\MakeUppercase{Software Engineer}} \\[0.1cm]
\textit{\textbf{TechCorp}, New York, NY, USA} \hfill \textit{Mar 2020 - Mar 2021}
\begin{itemize}[leftmargin=0.4cm]
    \item Worked as a software engineer at TechCorp, contributing to various software development projects.
\end{itemize}

% ----------------------------
% Add sub-sections for each job as follows (e.g. in case of consultancy/contractor experience)
\begin{tcolorbox}[colback=white, boxrule=0pt, enhanced, borderline west={1pt}{5pt}{accentcolor}, frame hidden]
    \textbf{\MakeUppercase{Data Analyst}} \\[0.1cm]
    \textit{\textbf{DataSolutions}} \hfill \textit{Mar 2020 - Dec 2020}
    \begin{itemize}[leftmargin=1cm]
        \item Conducted data analysis using Python, Excel, and MATLAB to identify trends and insights.
        \item Developed data visualization tools to present findings to stakeholders.
        \item Analyzed large datasets to support decision-making processes.
        \item Presented technical information in a clear and concise manner to clients.
    \end{itemize}
\end{tcolorbox}

\begin{tcolorbox}[colback=white, boxrule=0pt, enhanced, borderline west={1pt}{5pt}{accentcolor}, frame hidden]
    \textbf{\MakeUppercase{Quality Assurance Engineer}} \\[0.1cm]
    \textit{\textbf{Innovatech}} \hfill \textit{Dec 2020 - Mar 2021}
    \begin{itemize}[leftmargin=1cm]
        \item Conducted testing and quality assurance for software products.
        \item Developed and executed test cases to ensure software met quality standards.
        \item Identified and resolved software defects, improving product reliability.
        \item Collaborated with development teams to ensure seamless integration and deployment.
    \end{itemize}
\end{tcolorbox}
        \end{nthcolumn}

        \begin{nthcolumn}{1}
            \raggedright % Align the content to the left
            \textcolor{accentcolor}{\textbf{\large {Personal Information}}} \\[0.2cm]

\begin{tabular}{@{}m{0.5cm}@{}m{4cm}@{}}
    \faEnvelope & \email{john.doe@example.com} \\[0.2cm]
    \faMobile & \mobile{+1 (555) 123-4567} \\[0.2cm]
    \faLinkedin & \linkedin{john-doe} \\[0.2cm]
    \faBirthdayCake & \dateofbirth{January 1st, 2000} \\[0.2cm]
    \faMapMarker* & \residence{Mannheim, Germany} \\[0.2cm]
    \faFlag & \citizenship{United States} \\[0.2cm]
\end{tabular}
\\
            \vspace{0.1cm}
            \textcolor{accentcolor}{\textbf{\large {Skills}}} \\[0.1cm]

\begin{itemize}[left=0pt]
    \item \textit{Programming Languages:}
        \begin{itemize}
            \item Python
            \item Java
            \item C++
            \item JavaScript
        \end{itemize}
    \item \textit{Web Development:}
        \begin{itemize}
            \item HTML/CSS
            \item React
            \item Node.js
            \item Django
        \end{itemize}
\end{itemize}
            \vspace{0.1cm}
            \textcolor{accentcolor}{\textbf{\large {Language}}} \\[0.2cm]

\begin{itemize}[left=0pt]
    \item English | Native speaker
    \item German | Advanced (B2)
    \item French | Intermediate (B1)
\end{itemize}
        \end{nthcolumn}
    \end{paracol}

    \vspace{0.5cm} % Add some vertical space after the columns
    %\makebox[0.94\textwidth][c]{\rule{\textwidth}{1pt}} % Center the horizontal line and set thickness
    \vspace{0.5cm}
    
    \newpage % USE FOR splitting the document for pagination

    \columnratio{0.6, 0.4}
    \begin{paracol}{2} % Create a two-column layout
        \setlength{\columnsep}{20pt} % Set the column separator width
        %\setlength{\columnseprule}{1pt} % Add a vertical rule between the columns

        \begin{nthcolumn}{0} % Changed from 0 to 1 to ensure proper column usage
            \raggedright % Align the content to the left
            \textcolor{accentcolor}{\textbf{\large {Education}}} \\[0.2cm]

\textbf{M.Sc. \MakeUppercase{in Computer Science}} \\[0.1cm]
\textit{\textbf{Stanford University}, Stanford, CA, USA} \\ \textit{Sep 2019 - Jun 2021}
\begin{itemize} [leftmargin=0.4cm]
    \item Specialized in Artificial Intelligence and Machine Learning, with coursework in advanced algorithms, data structures, and neural networks.
    \item Conducted research on natural language processing and published a paper in a peer-reviewed journal.
\end{itemize}


\vspace{0.5cm}

\textbf{B.Sc. \MakeUppercase{in Computer Science}} \\[0.1cm]
\textit{\textbf{University of California, Berkeley}, Berkeley, CA, USA} \\ \textit{Aug 2013 - May 2017}
\begin{itemize} [leftmargin=0.4cm]
    \item Graduated with honors, focusing on software engineering, algorithms, and computer systems.
    \item Completed a senior project on distributed systems, which was presented at a national conference.
\end{itemize}

\vspace{0.5cm}

\textbf{\MakeUppercase{Study Abroad Program}} \\[0.1cm]
\textit{\textbf{University of Tokyo}, Tokyo, Japan} \\ \textit{Jan 2018 - Jul 2018}
\begin{itemize} [leftmargin=0.4cm]
    \item Participated in a six-month study abroad program focusing on advanced topics in computer science and engineering.
    \item Engaged in cross-cultural exchange and collaborative projects with international students.
\end{itemize}
        \end{nthcolumn}

        \begin{nthcolumn}{1}
            \raggedright % Align the content to the left
            \textcolor{accentcolor}{\textbf{\large {Relevant Coursework}}} \\[0.1cm]

\begin{itemize}[left=0pt]
    \item \textit{Computer Science}
    \begin{itemize}
        \item Data Structures and Algorithms
        \item Operating Systems
        \item Database Management Systems
        \item Computer Networks
    \end{itemize}
    \item \textit{Mathematics}
    \begin{itemize}
        \item Calculus I & II
        \item Linear Algebra
        \item Probability and Statistics
    \end{itemize}
    \item \textit{Engineering}
    \begin{itemize}
        \item Circuit Analysis
        \item Digital Logic Design
        \item Control Systems
    \end{itemize}
\end{itemize}

        \end{nthcolumn}
    \end{paracol}

\end{document}